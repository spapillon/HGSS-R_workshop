%% Notes or not
%%\documentclass[10pt,t,handout]{beamer}
%%%%
\documentclass[10pt]{beamer}
%%
\usepackage{pgfpages}
\usepackage{graphicx}
\usepackage{ulem}
\usepackage{color}
\usepackage{fancyvrb}
% get rid of junk
\usetheme{default}
\beamertemplatenavigationsymbolsempty
\hypersetup{pdfpagemode=UseNone} % don't show bookmarks on initial view
% font
\usefonttheme{professionalfonts}
\usefonttheme{serif}
% page number
\setbeamertemplate{footline}{%
    \raisebox{5pt}{\makebox[\paperwidth]{\hfill\makebox[20pt]{\color{gray}
          \scriptsize\insertframenumber}}}\hspace*{5pt}}
% add a bit of space at the top of the notes page
  \addtobeamertemplate{note page}{\setlength{\parskip}{12pt}}

\setbeamertemplate{blocks}[rounded][shadow=true]

% Notes or not
\setbeameroption{show notes}
%%\setbeamertemplate{note page}[plain]
%%\pgfpagesuselayout{2 on 1}
%%%%
%%\setbeameroption{hide notes}

\newenvironment{xframe}[2][]
  {\begin{frame}[fragile,environment=xframe,#1]
  \frametitle{#2}}
  {\end{frame}}

\title{Workshop: Introduction to R}
\author{Jean Monlong \& Simon Papillon}
\institute{Human Genetics department}
\date{October 21, 2013}

\begin{document}

%%%%%%%%%%%%%%%%%%%%
%% Title Slide
\begin{frame}
  \titlepage
  \centering
  \includegraphics[page=1,height=.1\textheight]{imgs/McGill-Logo1.png}

  \note{Welcome blabla, \\ 
Who's a complete beginner ? \\ 
Give me a R, give me a ... well that's it\\ 
Potential addition: matrix fill up scheme, Rstudio use guide/slide, Tips box, Question box\\
Exercises/data to prepare: \\nice/funny plots\\useful function\\debugging\\one-liner quiz.}
\end{frame}

%%%%
%%%%%%%%%%%%%%%%%%%%%%%
\section{Why learning R ?}
%%%%%%%%%%%%%%%%%%%%%%%
%%%%

\begin{frame}{Why learning R ?}
\begin{block}{Useful for your research}
  \begin{itemize}
  \item To explore your results. Curiosity and safety !
  \item To do/understand your analysis. Independence and control !
  \item To apply the latest Bioinformatics analyzes. Bioconductor !
  \item To keep track of your analysis. Reproducibility and automation !
    \bigskip
  \item You do it, not some busy bioinformatician. 
  \end{itemize}
\end{block}
\begin{block}{It's a good time investment}
  \begin{description}
  \item[Simple:] interpretative language(no compilation needed), no memory management, +++
  \item[Free:] widely used, vast community of R users, good life expectancy.
  \item[Multiplatform:] Windows, Mac, Unix, it works everywhere.
  \end{description}
\end{block}
\note{671 packages in Bioconductor. Bioconductor provides tools for the analysis and comprehension of high-throughput genomic data.}
\end{frame}

%%%%%%%

\begin{xframe}{Comparison to other languages}
  Comparison with C ?
  \note{The shuffle array example is good}
\end{xframe}

%%%%%%%

\begin{frame}{R and Rstudio}
  \begin{block}{Easy installation}
    \begin{itemize}
    \item Install R from  \\ \url{http://cran.r-project.org/}
    \item Install Rstudio Desktop from \\ \url{http://www.rstudio.com/ide/download/desktop}
    \end{itemize}
  \end{block}
  \centering
  \bigskip
  \includegraphics[page=1,height=.3\textheight]{imgs/Rstudio.png}

  \note{Emacs+ESS on Linux, R console on Mac}
\end{frame}



%%%%
%%%%%%%%%%%%%%%%%%%%%%%
\section{Data structures}
%%%%%%%%%%%%%%%%%%%%%%%
%%%%

\begin{frame}{Data structure - Overview}
  \begin{block}{Unit type}
    \begin{description}
      \item[numeric] Numbers, e.g. $0$, $1$, $42$, $-66.6$.
      \item[character] Words, e.g. ``male'', ``ENSG0007'',``Vive la France''.
      \item[logical] Binary, i.e. two possible values: {\it TRUE} or {\it FALSE}.
    \end{description}
  \end{block}
  \begin{columns}
    \begin{column}{.8\textwidth}
      \begin{block}{Structure}
        \begin{description}
        \item[vector] Ordered collection of elements of the same type.
        \item[matrix] Matrix of element of the same type.
        \item[list] Flexible container, mixed type possible. Recursive.
        %%\item[data.frame] Table-like structure, same type within a column.  Recursive.
        \end{description}
      \end{block}      
    \end{column}
    \begin{column}{.2\textwidth}
      \includegraphics[width=\linewidth]{imgs/vectorMatrixCartoon.png}
    \end{column}
  \end{columns}
\note{Other type but more complex and less useful, e.g. factors}
\end{frame}


\begin{xframe}{Assign a value to an object}
  \begin{block}{Choose an object name}
    \begin{itemize}
    \item {\bf Letters}, {\bf numbers}, {\bf dot} or {\bf underline} characters.
    \item {\bf Starts with a letter} or the dot not followed by a number.
    \item Correct: ``{\sf valid.name}'', "{\sf valid\_name}", "{\sf valid2name3}".
    \item Incorrect: "{\sf valid name}", "{\sf valid-name}", "{\sf 1valid2name3}".
    \end{itemize}
  \end{block}
  \begin{block}{Assign a value}
    The name of the object followed by the assignment symbol and the value.
    \medskip
\begin{Verbatim}[commandchars=\\\{\}]
\color{red}valid.name_123 \color{green!60!black}= \color{black}1
\color{red}valid.name_123 \color{green!60!black}<- \color{black}1

\color{red}valid.name_123
\end{Verbatim}
  \end{block}
\end{xframe}

%%% VECTORS
%%%%%%%%%%%%%%%%%%%%
\begin{xframe}{Vectors}
  \begin{block}{Vector construction}
    \begin{description}
    \item[c] Concatenate function.
    \item[1:10] Vector with numbers from 1 to 10.
    \item[rep] Repeat element several times.
    \end{description}
  \end{block}
  \begin{exampleblock}{Example}
\begin{verbatim}
luckyNumbers = c(4,8,15,16,23,42) 
luckyNumbers
oneToTen = 1:10
tenOnes = rep(1,10)
samples = c("sampA","sampB")
samples
\end{verbatim}
  \end{exampleblock}
  \begin{alertblock}{Everything is a vector}
    is.vector(is.vector(1)) $->$ TRUE
  \end{alertblock}
\note{Questions: Create your own numbers and favorite group of friends/hockey player/star/genes.}
\end{xframe}

%%%%%%%

\begin{xframe}[shrink=5]{Vectors}
  \begin{block}{Characterization}
    \begin{description}
    \item[length] Number of element in the vector.
    \item[names] Get or set the names of the vector.
    \end{description}
  \end{block}
  \begin{block}{Manipulation}
    \begin{description}
    \item[{vec[i:j]} ]Subset a vector from $i^{th}$ to $j^{th}$ values.
    \item[sort] Sort a vector.
    \item[order] Get the index of the sorted elements.
    \item[rev] Reverse a vector.
    \item[sample] Shuffle a vector.
    \end{description}
  \end{block}
  \begin{exampleblock}{Example}
\begin{verbatim}
length(luckyNumbers)
luckyNumbers[2:4]
names(luckyNumbers) = c("frank","henry","philip",
                            "steve","tom","francis") 
luckyNumbers
luckyNumbers["philip"]
rev(1:10)
order(c(luckyNumbers,1:10,tenOnes))
\end{verbatim}
  \end{exampleblock}
  \note{Square-brackets\\Questions: \\change the third number,\\ print a shuffle version of the vector \\add ``Jean'' at the end of the character vector, \\reverse it, \\make the reverse the new value.}
\end{xframe}

%%%%%%%

\begin{xframe}{Vectors}
  \begin{block}{Exploration}
    \begin{description}
    \item[head/tail] Print the first/last values.
      \medskip
    \item[{\bf\small On numeric vectors:}]
    \item[summary] Summary statistics: minimum, mean, maximum, ...
    \item[min/max/mean/var] Minimum, maximum, average, variance.
    \item[sum] Sum of the vector's values.
    \end{description}
  \end{block}
  \begin{exampleblock}{Example}
\begin{verbatim}
head(samples)
summary(luckyNumbers)
mean(luckyNumbers)
\end{verbatim}
  \end{exampleblock}
\note{Tips: na.rm\\Questions:\\Show me the beginning of your numbers\\the names of your numbers\\change the name of the second value to something\\average value of this beginning\\the sum of the minimum and maximum value.}
\end{xframe}

%%%%%%%

\begin{xframe}{Vectors}
  \begin{block}{Operations}
    \begin{itemize}
    \item Simple arithmetic operations over all the values of the vector.
    \item  Or values by values when using vectors of same length.
    \item Arithmetic operation: +, -, *, /.
    \end{itemize}
  \end{block}
\begin{exampleblock}{Example}
\begin{verbatim}
luckyNumbers * 4 - 2
luckyNumber s* 1:length(luckyNumbers) - 
                          rev(1:length(luckyNumbers))
\end{verbatim}
  \end{exampleblock}
  \note{Let's apply it to the Exercise}
\end{xframe}

%%%%%%%

\begin{frame}{Exercise - Guess my favorite number}
  \begin{block}{Instructions}
    \begin{enumerate}
    \item Create a vector of {\it numeric} values. At least two values.
    \item Multiply it by $6$.
    \item Add $21$.
    \item Divide it by $3$ 
    \item Remove $1$.
    \item Halve it.
    \item Remove its original values.
    \end{enumerate}
  \end{block}

  \note{Tips: save the original values somewhere or change the values of a new vector.}
\end{frame}


%%% MATRIX
%%%%%%%%%%%%%%%%%%%%

\begin{xframe}{Matrix}
  \begin{block}{Specific to matrices}
    \begin{description}
    \item[matrix] Create a matrix.
    \item[rbind/cbind] Concatenate vectors or matrix by row or column.
    \item[{mat[i:j,k:l]} ] Subset from the $i$ to $j$ row and $k$ to $l$ column.
    \item[dim] Dimension of the matrix: number of rows and columns.
    \item[rownames/colnames] Get or set the names of the rows/columns.
    \end{description}    
  \end{block}
  \begin{exampleblock}{Example}
\begin{verbatim}
mat = matrix(runif(12),3,4)
colnames(mat) = c("col1","col2","col3","col4")
rownames(mat) = c("row1","row2","row3")
\end{verbatim}
  \end{exampleblock}
  \note{Questions: \\create 4x4 matrix with number from 1 to 16\\ the same but shuffled\\print the first column\\ the three first columns \\ Add an extra line to the matrix\\Print the new dimension}
\end{xframe}

%%%%%%%

\begin{xframe}{Matrix}
  \begin{block}{Same as {\sf vector}}
    \begin{itemize}
    \item {\sf length}, {\sf head}, {\sf tail}.
    \item For numeric matrix: {\sf min}, {\sf max}, {\sf sum}, {\sf mean}.
    \item Arithmetic operations: +, -, *, /.
    \end{itemize}
  \end{block}
  \begin{exampleblock}{Example}
\begin{verbatim}
mean(mat)
sum(mat) / length(mat)

mat * 2
mat + mat
\end{verbatim}
  \end{exampleblock}
  \note{Questions:\\Average of the matrix\\Average of the first two columns\\ multiply by 2 and remove the matrix}
\end{xframe}

%%%%%%%

\begin{xframe}{Lists}
  \begin{block}{Flexible container}
    A list can contain any element type. It does not require elements to be of
    the same type.
    \begin{description}
      \item[list] Create a list.
      \item[{l[[i]]} ] Get or set the $i^{th}$ object of the list.
      \item[l\$toto] Get or set the element labeled as {\it toto}.
      \item[names] Get or set the names of the list elements.
      \item[length] Get the number of element in the list.
      \item[str] Output the structure of a R object.
    \end{description}
  \end{block}
  \begin{exampleblock}{Example}
\begin{verbatim}
l = list(vec=1:10,mat=matrix(runif(25),5))
str(l)
l
l$vec = 1
l
\end{verbatim}
  \end{exampleblock}
  \note{Questions:\\Make a phonebook: A list of 3 elements (vectors): names,
  phone number and address}
\end{xframe}

%%%%%%%

\begin{frame}{Exercise}
  \begin{enumerate}
  \item Create a matrix of with 100 rows and 4 columns with random numbers inside. {\scriptsize\it Tip: {\sf runif} function for random numbers.}
  \item Name the columns. E.g. {\it sampleA}, {\it sampleB}, ...
  \item Print the name of the column with the largest mean value.
  \item Print the name of the column with the largest value.
  \end{enumerate}
  \note{What if it had 100 rows...}
\end{frame}

%%%%%%%

\begin{xframe}{Functions - {\sf apply}}
  \begin{block}{New best friend}
    \begin{itemize}
    \item Apply a function to row or columns of a 2 dimension data structure (matrix or data frame).
    \item No manual iteration, the loop is implicit.
    \item Second argument: $1$ means rows, $2$ means columns.
    \end{itemize}
  \end{block}
  \begin{exampleblock}{Example}
\begin{verbatim}
apply(mat,1,mean)
apply(mat,2,function(x){
  x.mean = mean(x)
  return(x.mean+1)
})
\end{verbatim}  
  \end{exampleblock}
  \note{Same for list, etc\\output}
\end{xframe}

%%%%%%%

\begin{xframe}{Functions - {\sf lapply}}
  \begin{block}{apply for lists}
    \begin{itemize}
    \item Useful way to iterate through lists.
    \end{itemize}
  \end{block}
  \begin{exampleblock}{Example}
\begin{verbatim}
file_list <- read.files('.')
files_content <- lapply(file_list, function(file) \{
	data <- read.csv(file)
	#Do something with the data
	return(data)
\})
\end{verbatim}  
  \end{exampleblock}
\end{xframe}


%%%%%%%
\section{Functions}
%%%%%%%

\begin{xframe}{Functions}
  \begin{block}{}
    \begin{itemize}
    \item Name of the function with arguments between parenthesis.
    \item E.g. {\sf mean(x)}.
    \end{itemize}
  \end{block}
  \begin{block}{Do your own}
    \begin{itemize}
      \item[function] To define functions.
      \item All the object created within the function are temporary.
      \item[return] Define what will be returned by the function. 
    \end{itemize}
  \end{block}
  \begin{exampleblock}{Example}
\begin{verbatim}
almostMean = function(x){
  x.mean = mean(x)
  return(x.mean+1)
}
almostMean(0:10)
x.mean
\end{verbatim}
  \end{exampleblock}
  \note{Question:create a function that returns the power: pow <- function(base, exp) ...\\}
\end{xframe}

%%%%%%%


%%%%%%%
\section{Conditions and loops}
%%%%%%%
\begin{xframe}{Conditions}
  \begin{block}{Boolean}
    \begin{description}
    \item[{\it logical}] Binary data: {\it TRUE} or {\it FALSE}.
    \item[Numeric comparison] {\sf ==}, {\sf !=}, {\sf $>$}, {\sf $<$}, {\sf $>=$}, {\sf $<=$}.
    \item[Boolean operation] AND: \&, OR: $|$, NOT: !
    \item[which] Returns the index of the vectors with {\it TRUE} values.
    \item[any] Take a vector of {\it logical} and return {\it TRUE} if at least one value is {\it TRUE}.
    \item[\%in\%] Vectorized any. See example/supp material.
    \end{description}
  \end{block}
  \begin{exampleblock}{Example}
\begin{verbatim}
2 + 2 == 4
(2 < 3) & (3 != 1+2)
which(5:10 == 6)
any(9>1:10)
any(9>1:10 & 8<=1:10)
luckyNumbers[which(luckyNumbers %in% c(16,42,-66.6))]
\end{verbatim}  
  \end{exampleblock}
  \note{Is more details on logical rules necessary ?\\Question: write a function that filters out numbers: largerThan <- function(data, threshold) \{...\} }
\end{xframe}

%%%%%%%

\begin{xframe}{Testing conditions}
  \begin{block}{{\sf if else}}
    Test if a condition, if {\it TRUE} run some instruction, if {\it FALSE} something else (or nothing).
  \end{block}
  \begin{exampleblock}{Example}
\begin{verbatim}
if(length(luckyNumbers)>3){
  cat("Too many lucky numbers.\n")
  luckyNumbers = luckyNumbers[1:3]
} else if(length(luckyNumbers)==3){
  cat("Just enough lucky numbers.\n")
} else {
  cat("You need more lucky numbers.\n")
}
\end{verbatim}  
  \end{exampleblock}
  \note{Maybe more theoretical structure\\Question: write a function that filter number higher than 10}
\end{xframe}

%%%%%%%

\begin{xframe}{Loops}
  \begin{block}{{\sf for} loops}
    Iterate over the element of a container and run instructions.
\begin{verbatim}
for(v in vec){
...  Instruction
}
\end{verbatim}  
  \end{block}
  \begin{block}{{\sf while} loops}
    Run instructions as long as a condition is {\it TRUE}.
\begin{verbatim}
while( CONDITION ){
...  Instruction
}
\end{verbatim}  
  \end{block}
  \note{Question:}
\end{xframe}




%%%%%%%
\section{Import/export data}
%%%%%%%

\begin{frame}[shrink=5]{Import/export data}
  \begin{block}{Easy but important}
    \begin{itemize}
    \item What data structure is the more appropriate ? {\sf vector}, {\sf matrix} ?
    \item Does R read/write the file the way you want ?
    \item The extra arguments of the functions are your allies.
    \end{itemize}
  \end{block}
  \begin{block}{{\sf scan}}
    To read a {\sf vector} from a file with, for example, one value per line.
    \begin{description}
    \item[file=] the file name.
    \item[what=] the type of the argument gives the type of the values, e.g $1$, $''a''$.
    \item[sep=] the character that separate each value. By default, a white-space or end of line.
    \end{description}
  \end{block}
  \begin{block}{{\sf write}}
    To write a {\sf vector} from a file with one value per line.
    \begin{description}
    \item[vec] the vector to write.
    \item[file=] the file name.
    \item[sep=] the character that separate each value.
    \end{description}
  \end{block}
  \note{Questions: try to write on vector\\Then re-read it.}
\end{frame}

%%%%%%%

\begin{frame}[shrink=10]{Import/export data}
  \begin{block}{{\sf read.data}}
    To read a {\sf data.frame} from a multi-column file.
    \begin{description}
    \item[file=] the file name.
    \item[header=] {\it TRUE} use the first line for the column names. Default: {\it FALSE}.
    \item[as.is=] {\it TRUE} read the values as simple type, no complex type inference, {\bf recommended}. Default: {\it FALSE}. 
    \item[sep=] the character that separate each column. By default, a white-space or end of line.
    \end{description}
  \end{block}
  \begin{block}{{\sf write.data}}
    To write a {\sf data.frame} in a multi-column file.
    \begin{description}
    \item[df] the {\sf matrix} or {\sf data.frame} to write.
    \item[file=] the file name.
    \item[col.names=] {\it TRUE} print the column names in the first line. Default: {\it TRUE}.
    \item[row.names=] {\it TRUE} print the rows names in the first columns. Default: {\it TRUE}.
    \item[quote=] {\it TRUE} surround {\sf character} by quotes($''$). Default: {\it TRUE} $\rightarrow$ messy. 
    \item[sep=] the character that separate each column. By default, a white-space.
    \end{description}
  \end{block}
  \note{Questions: try to write a matrix with the different arguments\\Then re-read it.}
\end{frame}

%%%%%%%

\begin{xframe}{Import/export data}
  \begin{block}{R objects}
    \begin{description}
      \item[save] Save R objects into a file. Usual extension: {\it .RData}. {\sf file=} argument to specify file name.
      \item[save.image] Save the entire R environment.
      \item[load] Load R objects from a ({\it .RData}) file. {\sf verbose} to print the names of the objects loaded.
    \end{description}
  \end{block}
  \begin{exampleblock}{Example}
\begin{verbatim}
save(luckyNumbers, tenOnes, mat, file="uselessData.RData")
load(file="uselessData.RData")
load(file="dataForBasicPlots.RData",verbose=TRUE)
\end{verbatim}  
  \end{exampleblock}
  \note{Rstudio tips\\Questions: load data for next exercise.\\Save your objects if you want to...}
\end{xframe}

%%%%%%%
\section{Basic plotting}
%%%%%%%

\begin{frame}{Basic plotting}
  \begin{block}{}
    \begin{description}
      \item[hist] Plot the value distribution of a vector.
      \item[plot] Plot one vector against the other.
      \item[line] Same as plot but super-imposed to the existent one.
      \item[abline] Draw vertical/horizontal lines.
    \end{description}
  \end{block}
  \begin{block}{Common arguments}
    \begin{description}
      \item[main=] A title for the plot.
      \item[xlim=/ylim] A vector of size two defining the desired limit on the x/y axis.
      \item[xlab=/ylab=] A name for the x/y axis.
    \end{description}
  \end{block}
  \note{Questions: plot the prepared data(some funny shaped plots ?)\\Histogram with vertical line on the mean}
\end{frame}

%%%%%%%

%%%%%%%
\section{Extra exercises}
%%%%%%%

\begin{frame}{Debugging}
  \begin{block}{Instructions}
    \begin{enumerate}
    \item Open {\bf scriptToDebug.R} document.
    \item Run and debug it !
    \end{enumerate}
  \end{block}
  \note{Bugs: header load table, type read.table, parenthesis/brackets, infinite loop, NA in mean etc, operation different length, type coercion numeric character, non-unique (col)names, (global variable within function), apply rows returning matrix}
\end{frame}

\begin{frame}{One-liner quiz}
  \begin{block}{Instructions}
    Write R command to address each question. Only one-line command allowed. The shorter the better.
  \end{block}
  \begin{block}{Questions}
    \begin{enumerate}
    \item From a matrix of numeric, compute the proportion of columns with average value higher than 0.
    \item From a matrix of numeric, print the name of the columns with the highest value.
    \item From a matrix of numeric, print the rows with only positive values.
    \item 
    \end{enumerate}
  \end{block}
  \note{Find more questions.}
\end{frame}

%%%%%%%
\section{Miscellaneous}
%%%%%%%

\begin{xframe}{Type coercion.}
  \begin{block}{}
    \begin{itemize}
    \item Automatic conversion of an object to another type, e.g {\sf numeric}$\rightarrow${\sf character}, {\sf logical}$\rightarrow${\sf numeric}.
    \item Awareness for debugging.
    \item Useful sometimes.
    \end{itemize}
  \end{block}
  \begin{exampleblock}{Example}
\begin{verbatim}
is.numeric( c(1:10,"eleven") )

logical.vector = c(TRUE,TRUE,FALSE,TRUE,FALSE)
sum(logical.vector)
mean(logical.vector)
\end{verbatim}  
  \end{exampleblock}
  \note{Questions: How would you do it }
\end{xframe}

\begin{xframe}{{\sf character} operations}
  \begin{block}{}
    \begin{description}
      \item[paste] Paste several character into one.
      \item[grep] Search a pattern in a vector and return the index when matched.
      \item[grepl] Search a pattern in a vector and return {\it TRUE} if found.
      \item[strsplit] Split character into several.
    \end{description}
  \end{block}
  \begin{exampleblock}{Example}
\begin{verbatim}
sample.name = "Ob5cU8eN4mE"
file.name = paste("pathToYourDirectory/greatAnalysis-",
                                sample.name,".txt",sep="")

which(sample.names=="controlA" & sample.names=="controlB")
grep("control",sample.names)
\end{verbatim}  
  \end{exampleblock}
  \note{More details}
\end{xframe}

\begin{xframe}{Valid object name}
  \begin{block}{}
    \begin{itemize}
    \item {\bf Letters}, {\bf numbers}, {\bf dot} or {\bf underline} characters.
    \item {\bf Starts with a letter} or the dot not followed by a number.
    \item {\sf make.names} convert character into valid object names.
    \end{itemize}
  \end{block}
  \begin{exampleblock}{Example}
\begin{verbatim}
make.names(c("valid name","valid_name","valid.name",
             "valid-name","2.valid.name","x2.valid-name"))
\end{verbatim}  
  \end{exampleblock}
  \note{Should it be present in the beginning ?}
\end{xframe}



\end{document}
