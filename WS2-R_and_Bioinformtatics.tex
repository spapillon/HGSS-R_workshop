% Notes or not
%%\documentclass[10pt,t,handout]{beamer}
%%%%
\documentclass[10pt]{beamer}
%%
\usepackage{pgfpages}
\usepackage{graphicx}
\usepackage{ulem}
\usepackage{color}
\usepackage{fancyvrb}
% get rid of junk
\usetheme{default}
\beamertemplatenavigationsymbolsempty
\hypersetup{pdfpagemode=UseNone} % don't show bookmarks on initial view
% font
\usefonttheme{professionalfonts}
\usefonttheme{serif}
% page number
\setbeamertemplate{footline}{%
    \raisebox{5pt}{\makebox[\paperwidth]{\hfill\makebox[20pt]{\color{gray}
          \scriptsize\insertframenumber}}}\hspace*{5pt}}
% add a bit of space at the top of the notes page
  \addtobeamertemplate{note page}{\setlength{\parskip}{12pt}}

\setbeamertemplate{blocks}[rounded][shadow=true]

% Notes or not
\setbeameroption{hide notes}
%%\setbeamertemplate{note page}[plain]
%%\pgfpagesuselayout{2 on 1}
%%%%
%%\setbeameroption{hide notes}

\newenvironment{xframe}[2][]
  {\begin{frame}[fragile,environment=xframe,#1]
  \frametitle{#2}}
  {\end{frame}}

\title{Workshop: R and Bioinformatics}
\author{Jean Monlong \& Simon Papillon}
\institute{Human Genetics department}
\date{October 28, 2013}

\begin{document}

%%%%%%%%%%%%%%%%%%%%
%% Title Slide
\begin{frame}
  \titlepage
  \centering
  \includegraphics[page=1,height=.1\textheight]{imgs/McGill-Logo1.png}

  \note{Welcome blabla: \\ 
    Today not package tutorial but ideas/tips.
    Online tutorial list todo
  }
\end{frame}

%%%%
%%%%%%%%%%%%%%%%%%%%%%%
\section{Why using R for bioinformatics ?}
%%%%%%%%%%%%%%%%%%%%%%%
%%%%

\begin{frame}[label=handout]{Why using R for bioinformatics ?}
  \begin{itemize}
  \item Flexible statistics and data visualization software.
  \item Many packages and a vast community: Bioconductor.
  \item Simple and easy, compared at other computing languages.
  \end{itemize}
  \note{671 packages in Bioconductor. Bioconductor provides tools for the analysis and comprehension of high-throughput genomic data.}
\end{frame}

%%%%%%%

\begin{frame}[label=handout]{Goal of the workshop}
  \begin{itemize}
  \item Explain and demonstrate some key principles to do good bioinformatics.
  \item How to structure the analysis, how to write efficient script. 
  \item What to do and not to do...
    \bigskip
  \item This is {\bf NOT} a package tutorial but will point at useful resources.
  \end{itemize}
\end{frame}

%%%%%%%

%%%%%%%
\section{Reminder and updates}

\begin{xframe}{Functions}
  {\it\small Previously on HGSS workshops: }
  \begin{block}{}
    \begin{description}
      \item[function] To define functions.
      \item[return] Define what will be returned by the function. 
    \end{description}
    All the object created within the function are temporary.
    \bigskip
    \end{block}
    \begin{block}{Structure}    
\begin{Verbatim}[commandchars=\\\{\}]
\color{green!60!black}myFunctionName \color{blue}<- function(\color{green!60!black}input.obj1\color{blue},\color{green!60!black}second.input.obj \color{blue}) \{
\color{black}...
... Intructions on 'input.obj1' and 'second.input.obj'
...
\color{blue}return(\color{green!60!black}my.output.obj\color{blue})
\color{blue}\}

\color{black}myFunctionName(1,c(2,4,5))
\end{Verbatim}
  \end{block}
\end{xframe}

\begin{xframe}{Conditions}
  \begin{block}{Logical tests}
	
    \begin{description}
    \item[{\sf ==}] Are both values equal.
    \item[{\sf $>$ or $>=$}] Is the left value greater (greater or equal)
    than the right value.
	\item[{\sf $<$ or $<=$}] Is the left value smaller (smaller or equal) than
	the left value.
    \item[!] Is a NOT operator that negates the value of a test.
    \item[$|$] Is an OR operator used to combine logical tests. Returns TRUE if
    either are TRUE.
    \item[\&] Is an AND operator used to combine logical tests. Returns TRUE
    if both are TRUE
    \end{description}
  \end{block}
  \begin{exampleblock}{Example}
\begin{verbatim}
test <- 2 + 2 == 4    ## (TRUE)
!test                 ## (FALSE)
test & !test          ## (FALSE)
test | !test          ## (TRUE)
\end{verbatim}  
  \end{exampleblock}
  \note{Here I changed the $||$ and \&\& by the simple $|$ and \&. I know they should be used for vectorized test but it works and it's simpler for them to remember.}
\end{xframe}

%%%%%%%%%%%

\begin{xframe}{Conditions}
  \begin{block}{Boolean}
  Any logical tests can be vectorized (compare 2 {\sf vector}s).
    \begin{description}
    %\item[$\mid$] Is a OR operator for vectorized application.
    %\item[\&] Is an AND operator for vectorized application.
    \item[which] Returns the index of the {\sf vector}s with {\it TRUE} values.
    %%\item[any] Take a {\sf vector} of {\it logical} and return {\it TRUE} if at least one value is {\it TRUE}.
    %%\item[\%in\%] Vectorized any. See example/supp material.
    \end{description}
  \end{block}
  \begin{exampleblock}{Example}
\begin{verbatim}
5:8 == 6                         ## FALSE,TRUE,FALSE,FALSE
5:8 >= 6 & 5:8<=7                ## FALSE,TRUE,TRUE,FALSE

c(TRUE, TRUE) & c(TRUE, FALSE)   ## TRUE,FALSE
c(TRUE, FALSE) | c(FALSE, FALSE) ## TRUE,FALSE

which(5:10 == 6)                 ## 2
which(5:10 > 6)                  ## 3,4,5,6
\end{verbatim}  
  \end{exampleblock}
  \note{Question: write a function that :\\filters out numbers smaller than 3\\The same with the threshold  as a parameter: {\sf largerThan $<-$ function(data, threshold) \{...\}} }
\end{xframe}
%%%%%%%%%

\begin{frame}{Conditions - Exercise}
  \begin{block}{Exercise}
  Create a function that: 
  \begin{enumerate}
  \item remove values below $3$ from a {\sf vector}.
  \item  remove values below a specified threshold from a {\sf vector}.
  \end{enumerate}
  \end{block}

  \bigskip

  \begin{block}{For more advanced users}
    Have a look at these functions on {\it logical} vectors:
    \begin{itemize}
    \item any, $\%$in$\%$.
    \item sum, mean, table.
    \end{itemize}
  \end{block}

  \bigskip

  \begin{block}{Extra tips}
    \begin{itemize}
    \item Don't filter out, keep in.
    \item Use the boolean vector directly between $[~]$.
    \end{itemize}
  \end{block}
\end{frame}

%%%%%%%

%%%%%%%

\begin{xframe}{Testing conditions}
  \begin{block}{{\sf if else}}
    Test a condition, if {\it TRUE} run some instruction, if {\it FALSE} something else (or nothing).
\begin{verbatim}
if( Condition ){
...   Instructions
} 
\end{verbatim}  
  \end{block}
  \begin{exampleblock}{Example}
\begin{verbatim}
if(length(luckyNumbers)>3){
  cat("Too many lucky numbers.\n")
  luckyNumbers = luckyNumbers[1:3]
} else if(length(luckyNumbers)==3){
  cat("Just enough lucky numbers.\n")
} else {
  cat("You need more lucky numbers.\n")
}
\end{verbatim}  
  \end{exampleblock}
  \note{Question: write a function that classify the average expression of a vector into ``low'' for lower than 3, ``medium'' between 3 and 7, ``high'' greater than 7. }
\end{xframe}

\begin{xframe}{Loops}
  \begin{block}{{\sf for} loops}
    Iterate over the element of a container and run instructions.
\begin{verbatim}
for(v in vec){
...  Instruction
}
\end{verbatim}  
  \end{block}
  \begin{block}{{\sf while} loops}
    Run instructions as long as a condition is {\it TRUE}.
\begin{verbatim}
while( CONDITION ){
...  Instruction
}
\end{verbatim}  
  \end{block}
  \begin{exampleblock}{Example}
\begin{verbatim}
facto = 1
for(n in 1:10){
   facto = facto * n
}
\end{verbatim}  
  \end{exampleblock}
  \note{Colour in structures\\Apply versus loop speech then vote if they want to do the exercice?}
\end{xframe}

%%%%%%%%%

\begin{frame}{Exercises}
  \begin{block}{{\sf if else}}
    Create a function that classify the average value of a {\sf vector}. It returns:
    \begin{itemize}
    \item {\it low} if the average if below $3$.
    \item {\it medium} if the average if between $3$ and $7$.
    \item {\it high} if the average if above $7$.
    \end{itemize}
  \end{block}
  
  \bigskip
  
  \begin{block}{Loops}
    Write a function that computes the mean values of a matrix columns:
    \begin{enumerate}
    \item using the {\sf apply}  function.
    \item using a {\sf for} loop.
    \item (using a {\sf while} loop.)
    \end{enumerate}
  \end{block}
\end{frame}


\section{Scripting and analysis structure}

\begin{frame}{Important principles}
  \begin{block}{Scripting}
    Write scripts of your analysis: 
    \begin{itemize}
    \item Keeping track, easy rerun, easy parameter tweaking.
    \item {\sf Rstudio} or other interfaces ({\sf Emacs+ESS},...).
    \end{itemize}
  \end{block}
  \begin{block}{Clear and modular code}
    \begin{itemize}
    \item Define clear analysis steps.
    \item Write function(s) for each step.
      \begin{itemize}
      \item Keeps the data and parameters used clear (for you and for R).
      \item No confusing temporary objects.
      \item No repeating code.
      \item More suitable for {\sf apply}-like usage.
      \item Easy parameter tweaking.
      \end{itemize}
    \end{itemize}
  \end{block}
  \begin{block}{Efficiency matters}
    \begin{itemize}
    \item Data structure and manipulation.
    \item Especially relevant with our large data.
    \end{itemize}

  \end{block}
  \note{Simon what interface do you use ?}
\end{frame}

\begin{frame}{Example}
  
\end{frame}

\begin{frame}{Big exercise}
  
\end{frame}

\section{Big Data Manipulation}

\begin{frame}{Memory and reading files}
  
\end{frame}

\begin{frame}{Memory and manipulating files}
  
\end{frame}




%%%%%%%%%%%
%%%%%%%%%%%%%%%%%%%%%
\section{Data exploration}
%%%%%%%%%%%%%%%%%%%%%
%%%%%%%%%%%

\begin{frame}{Data exploration - All you can plot}
  \begin{block}{Utility}
    \begin{itemize}
    \item Get an idea of the quality of the data and potential issues.
    \item Get a full answer.
    \item Detect potential biases.
    \item (Find unexpected results.)
    \end{itemize}
  \end{block}
  
  \begin{block}{Through all your analysis}
    \begin{itemize}
    \item Quality Control plots at the beginning.
    \item Control plot after each steps.
    \item Awesome plot with your results.
    \end{itemize}
  \end{block}
\end{frame}

\begin{frame}{QC plots}
  \begin{block}{Aim}
    Assess the quality of your data and potential artefacts that could bias your analysis.
  \end{block}
  \begin{block}{Basic approaches}
    \begin{itemize}
    \item Principal Component Analysis: representing the largest variation in the data.
    \item Clustering: summarizing similarity relations between samples/genes.
    \item Testing metadata: gender, age, ...
    \end{itemize}
  \end{block}
  \note{wikipedia cartoon for PCA, cluster tree}
\end{frame}

\begin{xframe}[shrink=10]{QC plots - Functions}
  \begin{block}{PCA using {\sf prcomp}}
    PCA of the matrix columns; plot of the variance explained by the first PCs; representation of the {\bf rows} using the first two PCs.
\begin{verbatim}
peeSeaAye = prcomp(input.matrix)
plot(peeSeaAye)
plot(peeSeaAye$x,type="n")
text(peeSeaAye$x,labels=rownames(input.matrix))
\end{verbatim}  
  \end{block}
  \begin{block}{Clustering using correlation distance}
    Clustering using a \uline{distance} matrix, e.g. from correlation between {\bf columns}.
\begin{verbatim}
cor.dist = as.dist(1-cor(input.matrix))

kleusteur = hclust(cor.dist,method="ward")
plot(kleusteur)

library(MASS)
mDeeS = isoMDS(cor.dist)
plot(mDeeS$points)
\end{verbatim}  
  \end{block}
  \note{highlight the functions\\Importance of the distance demonstration}
\end{xframe}

\begin{frame}{Exercise/Example}
  Do it on previous exercice
\end{frame}

\begin{frame}{Heatmaps}
  
\end{frame}

\begin{frame}{Testing bias from metadata}
  
\end{frame}

\begin{frame}{Extra ggplot2 package}
  
\end{frame}

\section{More resource online}

\begin{frame}{Online tutorials}
  
\end{frame}

\begin{frame}{Useful packages}
  
\end{frame}

\end{document}
