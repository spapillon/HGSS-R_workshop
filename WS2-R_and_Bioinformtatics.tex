% Notes or not
%%\documentclass[10pt,t,handout]{beamer}
%%%%
\documentclass[10pt]{beamer}
%%
\usepackage{pgfpages}
\usepackage{graphicx}
\usepackage{ulem}
\usepackage{color}
\usepackage{fancyvrb}
% get rid of junk
\usetheme{default}
\beamertemplatenavigationsymbolsempty
\hypersetup{pdfpagemode=UseNone} % don't show bookmarks on initial view
% font
\usefonttheme{professionalfonts}
\usefonttheme{serif}
% page number
\setbeamertemplate{footline}{%
    \raisebox{5pt}{\makebox[\paperwidth]{\hfill\makebox[20pt]{\color{gray}
          \scriptsize\insertframenumber}}}\hspace*{5pt}}
% add a bit of space at the top of the notes page
  \addtobeamertemplate{note page}{\setlength{\parskip}{12pt}}

\setbeamertemplate{blocks}[rounded][shadow=true]

% Notes or not
\setbeameroption{hide notes}
%%\setbeamertemplate{note page}[plain]
%%\pgfpagesuselayout{2 on 1}
%%%%
%%\setbeameroption{hide notes}

\newenvironment{xframe}[2][]
  {\begin{frame}[fragile,environment=xframe,#1]
  \frametitle{#2}}
  {\end{frame}}

\title{Workshop: R and Bioinformatics}
\author{Jean Monlong \& Simon Papillon}
\institute{Human Genetics department}
\date{October 28, 2013}

\begin{document}

%%%%%%%%%%%%%%%%%%%%
%% Title Slide
\begin{frame}
  \titlepage
  \centering
  \includegraphics[page=1,height=.1\textheight]{imgs/McGill-Logo1.png}

  \note{Welcome blabla: \\ 
    Today not package tutorial but ideas/tips.
    Online tutorial list todo
  }
\end{frame}

%%%%
%%%%%%%%%%%%%%%%%%%%%%%
\section{Why using R for bioinformatics ?}
%%%%%%%%%%%%%%%%%%%%%%%
%%%%

\begin{frame}[label=handout]{Why using R for bioinformatics ?}
  \begin{itemize}
  \item Flexible statistics and data visualization software.
  \item Many packages and a vast community: Bioconductor.
  \item Simple and easy, compared at other computing languages.
  \end{itemize}
  \note{671 packages in Bioconductor. Bioconductor provides tools for the analysis and comprehension of high-throughput genomic data.}
\end{frame}

%%%%%%%

\begin{frame}[label=handout]{Goal of the workshop}
  \begin{itemize}
  \item Explain and demonstrate some keys to do good bioinformatics.
  \item How to structure the analysis, how to write efficient script. 
  \item What to do or not to do...
    \bigskip
  \item This is {\bf NOT} a package tutorial but will point at useful resources.
  \end{itemize}
\end{frame}

%%%%%%%

%%%%%%%
\section{Reminder and updates}

\begin{xframe}{Functions}
  {\it\small Previously on HGSS workshops: }
  \begin{block}{}
    \begin{description}
      \item[function] To define functions.
      \item[return] Define what will be returned by the function. 
    \end{description}
    All the object created within the function are temporary.
    \bigskip
    \end{block}
    \begin{block}{Structure}    
\begin{Verbatim}[commandchars=\\\{\}]
\color{green!60!black}myFunctionName \color{blue}<- function(\color{green!60!black}input.obj1\color{blue},\color{green!60!black}second.input.obj \color{blue}) \{
\color{black}...
... Intructions on 'input.obj1' and 'second.input.obj'
...
\color{blue}return(\color{green!60!black}my.output.obj\color{blue})
\color{blue}\}

\color{black}myFunctionName(1,c(2,4,5))
\end{Verbatim}
  \end{block}
\end{xframe}

\begin{xframe}{Conditions}
  \begin{block}{Logical tests}
	
    \begin{description}
    \item[{\sf ==}] Are both values equal.
    \item[{\sf $>$ or $>=$}] Is the left value greater (greater or equal)
    than the right value.
	\item[{\sf $<$ or $<=$}] Is the left value smaller (smaller or equal) than
	the left value.
    \item[!] Is a NOT operator that negates the value of a test.
    \item[$|$] Is an OR operator used to combine logical tests. Returns TRUE if
    either are TRUE.
    \item[\&] Is an AND operator used to combine logical tests. Returns TRUE
    if both are TRUE
    \end{description}
  \end{block}
  \begin{exampleblock}{Example}
\begin{verbatim}
test <- 2 + 2 == 4    ## (TRUE)
!test                 ## (FALSE)
test & !test          ## (FALSE)
test | !test          ## (TRUE)
\end{verbatim}  
  \end{exampleblock}
  \note{Here I changed the $||$ and \&\& by the simple $|$ and \&. I know they should be used for vectorized test but it works and it's simpler for them to remember.}
\end{xframe}

%%%%%%%%%%%

\begin{xframe}{Conditions}
  \begin{block}{Boolean}
  Any logical tests can be vectorized (compare 2 {\sf vector}s).
    \begin{description}
    %\item[$\mid$] Is a OR operator for vectorized application.
    %\item[\&] Is an AND operator for vectorized application.
    \item[which] Returns the index of the {\sf vector}s with {\it TRUE} values.
    %%\item[any] Take a {\sf vector} of {\it logical} and return {\it TRUE} if at least one value is {\it TRUE}.
    %%\item[\%in\%] Vectorized any. See example/supp material.
    \end{description}
  \end{block}
  \begin{exampleblock}{Example}
\begin{verbatim}
5:8 == 6                         ## FALSE,TRUE,FALSE,FALSE
5:8 >= 6 & 5:8<=7                ## FALSE,TRUE,TRUE,FALSE

c(TRUE, TRUE) & c(TRUE, FALSE)   ## TRUE,FALSE
c(TRUE, FALSE) | c(FALSE, FALSE) ## TRUE,FALSE

which(5:10 == 6)                 ## 2
which(5:10 > 6)                  ## 3,4,5,6
\end{verbatim}  
  \end{exampleblock}
  \note{Question: write a function that :\\filters out numbers smaller than 3\\The same with the threshold  as a parameter: {\sf largerThan $<-$ function(data, threshold) \{...\}} }
\end{xframe}
%%%%%%%%%

\begin{frame}{Conditions - Exercise}
  \begin{block}{Exercise}
  Create a function that: 
  \begin{enumerate}
  \item remove values below $3$ from a {\sf vector}.
  \item  remove values below a specified threshold from a {\sf vector}.
  \end{enumerate}
  \end{block}

  \vspace{.2\textheight}

  \begin{block}{For more advanced users}
    Have a look of these functions on {\it logical} vectors:
    \begin{itemize}
    \item any, $\%$in$\%$.
    \item sum, mean, table.
    \end{itemize}
  \end{block}
\end{frame}

%%%%%%%

%%%%%%%

\begin{xframe}{Testing conditions}
  \begin{block}{{\sf if else}}
    Test if a condition, if {\it TRUE} run some instruction, if {\it FALSE} something else (or nothing).
\begin{verbatim}
if( Condition ){
...   Instructions
} 
\end{verbatim}  
  \end{block}
  \begin{exampleblock}{Example}
\begin{verbatim}
if(length(luckyNumbers)>3){
  cat("Too many lucky numbers.\n")
  luckyNumbers = luckyNumbers[1:3]
} else if(length(luckyNumbers)==3){
  cat("Just enough lucky numbers.\n")
} else {
  cat("You need more lucky numbers.\n")
}
\end{verbatim}  
  \end{exampleblock}
  \note{Question: write a function that classify the average expression of a vector into ``low'' for lower than 3, ``medium'' between 3 and 7, ``high'' greater than 7. }
\end{xframe}

%%%%%%%%%

\begin{frame}{{\sf if else} - Exercise}
  Create a function that classify the average value of a {\sf vector}. It returns:
  \begin{itemize}
  \item {\it low} if the average if below $3$.
  \item {\it medium} if the average if between $3$ and $7$.
  \item {\it high} if the average if above $7$.
  \end{itemize}
\end{frame}


\section{Scripting and analysis structure}

\begin{frame}{Principles}
  Scripts, softwares
  Functions/modules: No repetition, understandable, well defined and optimized(memory, data structure).
\end{frame}

\begin{frame}{Example}
  
\end{frame}

\begin{frame}{Big exercise}
  
\end{frame}

\section{QC plots}

\begin{frame}{Introduction}
  At the beginning, overview data, cluster, pca
\end{frame}

\begin{frame}{Functions}
  
\end{frame}

\begin{frame}{Exercise/Example}
  
\end{frame}

\section{Big Data Manipulation}

\begin{frame}{Memory and reading files}
  
\end{frame}

\begin{frame}{Memory and manipulating files}
  
\end{frame}

\section{Data exploration}

\begin{frame}{All you can plot}
  
\end{frame}

\begin{frame}{Extra ggplot2 package}
  
\end{frame}

\section{More resource online}

\begin{frame}{Online tutorials}
  
\end{frame}

\begin{frame}{Useful packages}
  
\end{frame}

\end{document}