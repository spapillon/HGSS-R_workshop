\documentclass[12pt]{article}

%%%%%%%%%%%%%%%%%%%
\usepackage[english]{babel}
\usepackage[utf8]{inputenc}
\usepackage{rotating}
\usepackage{graphicx}
\usepackage[colorlinks=true]{hyperref}
\usepackage{amsmath, amssymb, amsthm}
\usepackage{listings}
\usepackage{color}
\usepackage{multirow}
\usepackage{array}
\usepackage{ulem}

%---%
\textheight 26cm
\textwidth 16cm
%\oddsidemargin 0cm
%\evensidemargin 0cm
\topmargin -20mm
\hoffset -15mm
\voffset -0mm
%---%               

%
\title{R (Practical) Commands Summary}
\author{\tiny HGSS Workshop 2013}
\date{}

\begin{document}
\maketitle

%%%%
%%%%%%%%%%%%%%%%%%%%%%%%%%%%
%%%%
\paragraph{R control}
\begin{itemize} 
  \item Quit R; Load a library; Get help about a function. \\
    \verb!q(); library(theLibraryYouWant); help(plot); ?plot!
  \item Comment code. \\
    \verb!## This is a comment, it won't be interpreted by R!
\end{itemize}

%%%%
%%%%%%%%%%%%%%%%%%%%%%%%%%%%
%%%%
\paragraph{Data import/export}
\begin{itemize} 
  \item Save/load R objects. \\
    \verb!save(study.res1, study,res2,file="file.RData")! \\
    \verb!load("file.RData",verbose=TRUE)!
  \item Write/read a table in a file. \\ 
    \verb!write.table(t,file="file.txt",quote=FALSE,sep="\t",row.names=FALSE)! \\
    \verb!read.table("file.txt",as.is=TRUE,header=TRUE)! 
  \item Write/read a vector in a file. \\ 
    \verb!write(vec,"file.txt",sep="\n")! \\
    \verb!scan(file="file.txt",what="a"); scan(vec,file="file.txt",what=1)!
\end{itemize}


%%%%
%%%%%%%%%%%%%%%%%%%%%%%%%%%%
%%%%
\paragraph{Data creation}
\begin{itemize}
  \item Concatenate into a vector. \\ 
    \verb!c(1,10,44); c("toto","tata","titi"); c(TRUE, TRUE, FALSE)!
  \item Vector from $1$ to $10$; with ten ones; from $1$ to $10$ every $0.5$. \\ 
    \verb!1:10; rep(1,10); seq(1,10,0.5)!
  \item Create matrix, e.g with numbers from $1$ to $12$ with $3$ rows and $4$ columns\\
    \verb!matrix(1:12,3,4)!
  \item Bind matrices on their row/columns. \\
    \verb!rbind(matrix1, matrix2); cbind(matrix1, matrix2)!
  \item Create list. \\
    \verb!list(elt1=1:10, anotherElt=matrix(1:12,3,4))!
  \item Create a data.frame.\\
    \verb!data.frame(oneToTen=1:10,elevenToTwenty=11:20)!
\end{itemize}


\newpage
%%%%
%%%%%%%%%%%%%%%%%%%%%%%%%%%%
%%%%
\paragraph{Data characterization} 
\begin{itemize} 
  \item Display information about the object structure\\ 
    \verb!str(mat); str(veryComplexList)!
  \item Length of a vector; dimension of a matrix or individually the number of rows and columns.\\
    \verb!length(vec); dim(mat); nrow(mat); ncol(mat)!
  \item Get or set the names of the vector's values; or matrix rows/columns.\\
    \verb!names(vec); rownames(mat); colnames(mat)! \\
    \verb!names(vec) = c("one","two","three")!
\end{itemize}


%%%%
%%%%%%%%%%%%%%%%%%%%%%%%%%%%
%%%%
\paragraph{Data exploration} 
\begin{itemize} 
\item Print the beginning/end of an object.\\ 
  \verb!head(vec); head(mat); tail(vec); tail(vec)!
\item Access element from vector, matrix or list.\\ 
  \verb!vec[1]; vec[2:4]; mat[1,1]; mat[2:4,6:7]; list[[1]]!
\item Display summary statistics of a vector or matrix.\\ 
  \verb!summary(vec); summary(mat)!
\item Print the sum, mean, minimum, maximum value of a vector or matrix.\\ 
  \verb!sum(vec); mean(mat); minimum(vec); maximum(mat)!
\item Same with the median, variance, standard deviation.\\ 
  \verb!median(vec); var(vec); sd(vec) !
\item Count how many times values are repeated in a vector.\\ 
  \verb!table(vec)!
\item Correlation between two vector of the \uline{same length}, or the columns of a matrix.\\ 
  \verb!cor(vec1,vec2); cor(mat)!
\end{itemize}


%%%%
%%%%%%%%%%%%%%%%%%%%%%%%%%%%
%%%%
\paragraph{Data manipulation}
\begin{itemize} 
  \item sort a vector in increasing or decreasing order \\ 
    \verb!sort(vec); sort(vec, decreasing=TRUE)!
  \item Reverse a vector; shuffle a vector. \\ 
    \verb!rev(vec); sample(vec)!
  \item Apply a function on a matrix row or columns; or on every element of a list. \\ 
    \verb!apply(mat,1,mean); apply(mat,2,mean); lapply(l,mean)!
  \item Subset the content of a data frame using conditions on its columns. \\ 
    \verb!subset(df, oneToTen > 5)!
\end{itemize}

\newpage


%%%%
%%%%%%%%%%%%%%%%%%%%%%%%%%%%
%%%%
\paragraph{Boolean} 
\begin{itemize} 
  \item {\it a} equal {\it b} ? different ? greater ? greater or equal ? \\ 
    \verb?a==b; a!=b; a > b; a >= b?
  \item Get the index satisfying a condition \\ 
    \verb?which(Condition)?
  \item Is there any {\it TRUE} values in the vector. \\ 
    \verb?any(vec)?
\end{itemize}



%%%%
%%%%%%%%%%%%%%%%%%%%%%%%%%%%
%%%%
\paragraph{Condition and loops}
\begin{itemize} 
  \item Test conditions
\begin{verbatim}
if(Condition){
   ...Instructions...
} else if(Condition2){
   ...Instructions2...
} else {
   ...Instructions3...
}
\end{verbatim}
  \item Loop over the values of a vector.
\begin{verbatim}
for(v in vec){
   ...Instructions...
} 
\end{verbatim}
  \item Loop until a condition is satisfied.
\begin{verbatim}
while(Condition){
   ...Instructions...
}
\end{verbatim}
\end{itemize}


\newpage
%%%%
%%%%%%%%%%%%%%%%%%%%%%%%%%%%
%%%%
\paragraph{Simple plotting} 
\begin{itemize} 
  \item Plot the distribution of a vector's values. \\ 
    \verb!hist(vec,main="Your title",xlab="x-axis label")!
  \item Plot one vector against another of the same length. \\ 
    \verb!plot(vec1,vec2,main="Your title",xlab="x-axis label",ylab="y-axis label")!
  \item Parameters for the style of the point or line, or the colour \\ 
    \verb!plot(... , pch=2, lty=3, col=4)!
  \item Superimpose horizontal/vertical lines to the existing plot. \\ 
    \verb!abline(h=0,lty=2); abline(v=2,lty=2)!
  \item Superimpose another plot to the existing plot. \\ 
    \verb!lines(vec3,vec4,type="p")!
  \item Open a connection to a pdf/png file; close it.
\begin{verbatim}
pdf("myPlot.pdf"); png("myPlot.png")
plot(...)
dev.off()
\end{verbatim}
\end{itemize}


\paragraph{Appendix}
\begin{itemize} 
  \item Tabulation and End Of Line character \\ 
    \verb!"\t"; "\n"!
\end{itemize}

\let\thefootnote\relax\footnote{\href{mailto:jean.monlong@mail.mcgill.ca}{jean.monlong@mail.mcgill.ca}}

\end{document}

